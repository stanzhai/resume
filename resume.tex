%!TEX program = xelatex
\documentclass[UTF8,margin,line]{res}
\usepackage{fontspec, xunicode, xltxtra, color, url}  
\setmainfont{Hiragino Sans GB}

\oddsidemargin  -0.5in
\evensidemargin -0.5in
\textwidth = 6.0in
\itemsep   = 0.0in
\parsep    = 0.0in

\newenvironment{list1}{
  \begin{list}{\ding{113}}{%
      \setlength{\itemsep}{0in}
      \setlength{\parsep}{0in} \setlength{\parskip}{0in}
      \setlength{\topsep}{0in} \setlength{\partopsep}{0in}
      \setlength{\leftmargin}{0in}}}{\end{list}}

\begin{document}

\name{翟士丹 (https://github.com/stanzhai)\vspace*{.1in}}

\begin{resume}

\section{\sc 自我介绍}
\textbf{主攻大数据(3年+);全栈工程师;} Apache Spark Contributor,对Spark一定研究;熟悉SQL及Web技术栈;热爱编码,喜欢探索未知领域。

\section{\sc 教育经历}
\textbf{曲阜师范大学},软件工程,2008-2012,本科

\section{\sc 主要技能}
\textbf{方向:} 大数据平台、分布式存储/计算;后端开发;基础平台 \\
\textbf{技术栈:} Spark、Hive、Hadoop、Nodejs、Docker、OpenResty \\
\textbf{编程语言:} Scala、Python、Go ... \\
\textbf{编辑器:} Vim

\section{\sc 工作经历}
\textbf{海致:}技术经理,2014.12,BDP大数据平台负责人,BDP核心研发 \\
\textbf{畅游:}研发组长,2012.07,实现游戏数据分析平台;爬虫、舆情分析系统;信息安全审计系统 \\
\textbf{浪潮:}实习,2011.12,C++军工项目研发

\section{\sc 项目经验}
\textbf{BDP:}底层数据平台实现及性能优化,Spark,2014-至今 \url{https://www.bdp.cn/home.html} \\
* 基于Spark实现通用查询及建模服务,为BDP提供基础计算能力 \\
* Spark框架性能优化及Bug修复(Spark 0.9 -> 2.2) \\
* 实现Spark任务过载保护及自愈机制 \\
* 设计并参与实现多级查询缓存模块 \\
* 设计并实现支持Update、Delete的数据接入服务 \\
* 基于分区表实现表模型增量执行机制,改进超大表建模速度 \\
* 基于Go实现基于优先级及Quota机制的查询调度模块 \\
* 参与实现并行DAG任务调度系统 \\
* 基于Node.js实现数据管理平台,用于BDP服务监控及问题排查 \\
* WIP: 基于Structured Streaming实现通用流式计算平台 \\

\section{\sc 其他}
\textbf{知乎:} \url{https://zhuanlan.zhihu.com/bigdata-spark} \\
\textbf{Blog: } \url{http://www.cnblogs.com/jasondan/} \\

\end{resume}
\end{document}
