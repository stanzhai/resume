%!TEX program = xelatex
\documentclass[UTF8,margin,line]{res}
\usepackage{fontspec, xunicode, xltxtra, color, url}  
\setmainfont{Hiragino Sans GB}

\oddsidemargin  -0.5in
\evensidemargin -0.5in
\textwidth = 6.0in
\itemsep   = 0.0in
\parsep    = 0.0in

\newenvironment{list1}{
  \begin{list}{\ding{113}}{%
      \setlength{\itemsep}{0in}
      \setlength{\parsep}{0in} \setlength{\parskip}{0in}
      \setlength{\topsep}{0in} \setlength{\partopsep}{0in}
      \setlength{\leftmargin}{0in}}}{\end{list}}

\begin{document}

\name{翟士丹 (mail@stanzhai.site)\vspace*{.1in}} 

\begin{resume}

\section{\sc 自我介绍}
\textbf{8年+工作经验,现居北京} \\
专注大数据领域,Apache Spark Contributor,有丰富的大数据BI系统及中台构建经验; \\
熟悉SQL、Web技术栈及Linux;工具控;擅长系统架构设计,热爱编码,喜欢探索未知领域; \\
12年开发经验,5年研发团队管理经验(90人)。

\section{\sc 教育经历}
\textbf{曲阜师范大学} 软件工程,2008-2012,本科

\section{\sc 职业相关}
\textbf{方向:} 图数据库、大数据、高性能计算;后端开发;基础平台 \\
\textbf{技术栈:} Spark、HBase、Hadoop、Spring、Docker、Nodejs \\
\textbf{编程语言:} Rust、Java、Scala、Python、Go ... \\
\textbf{编辑器:} Vim

\section{\sc 工作经历}
\textbf{海致星图:}副总裁,2020.09,分布式图数据库AtlasGraph;知识图谱分析平台Atlas \\
\textbf{海致网络:}技术副总监、架构师,2014.12,SaaS BI(BDP);数据中台产品 \\
\textbf{搜狐·畅游:}研发组长,2012.07,通用数据分析平台;爬虫、舆情分析系统;信息安全审计系统 \\
\textbf{浪潮:}实习,2011.12,空军装备研究院,作战指挥系统(C++) \\
\textbf{点击科技:}实习,2010.07,LavaLava即时通讯iOS客户端开发(Objective-C) \\
\textbf{国软:}实习,2009.05,Flex棋盘类游戏逻辑开发(ActionScript、Lua) \\

\section{\sc 项目经验}
\textbf{智能中台(数据+计算+业务)服务公安、金融} Spark,ES,JanusGraph,GeoMesa 2018-至今 \\
- 基于标准SQL抽象了算子体系(关联、聚合、追加等),实现了高度灵活的可视化数据建模系统 \\
- 基于Structured Streaming实现通用流式计算平台(支持Kafka、可视化拖拽式建模) \\
- 实现主题库(对应数仓的DM层)管理模块,支持行业业务抽象的封装 \\
- 实现通用标签计算引擎,为特定对象自定义标签规则计算标签,支持知识图谱应用 \\
- 关系库设计实现 \\
- 异构任务调度平台构建 \\
- 业务算子系统开发,数据发布接口实现 \\
- 机器学习平台构建 \\
- 知识图谱分析平台实现 \\
- GIS时空分析平台实现 \\
\\
\textbf{BDP商业数据分析平台} Spark, BI 2014-2018 \url{https://www.bdp.cn} \\
参与了一站式大数据分析平台BDP产品从0到1的构建,支撑了数千家企业客户、万亿级规模数据分析服务。 \\
- 基于Spark实现高性能、任意多维度分析的OLAP查询引擎(每日20W查询任务,平均2s) \\
- Spark性能优化、稳定性改进及Bug修复,任务超时、过载保护及自愈机制(Spark 0.9 -> 2.3) \\
- 基于分区表实现MaterialView,支持增量更新,改进了千亿级以上Hive表建模及查询速度 \\
- 扩展了SQL语法,SQL层面支持机器学习处理流程、DataFrame源码层面的任意数据处理 \\
- 重构了Hive UDF在Spark的注册机制,实现UDF随应用启动自注册,极大简化了开发上线流程 \\
- 自研Patch算法,基于追加写实现表数据的Insert、Update、Delete的数据接入服务 \\
- 实现基于版本的并行DAG数据建模任务调度系统(日均调度任务50W) \\
- 基于Go实现基于优先级及Quota机制的查询调度模块(高性能请求转发、低内存占用) \\
- 基于PhantomJS实现高质量图表、仪表盘后端渲染导出服务 \\
- 构建服务端Sandbox环境,实现用户自定义图表JS代码的异常检测(死循环代码、超内存消耗) \\
- 基于Redis实现多级查询缓存模块,改进了查询缓存命中率,优化图表查询体验 \\
- 实现了BDP数据平台底层无缝切换至阿里云平台(HDFS+Spark -> ODPS+ADS) \\
- 基于Node.js+Knockout实现运维管理平台(数据血缘,SQL IDE,任务监控等) \\
\\
\textbf{MOPlatform数据分析监控平台} ASP.NET MVC4, RESTful, MVVM 2013/10-2014/12 \\
- 建立高度自定义的报表模块,用于数据可视化,通过灵活的报表配置,自定义报表内容 \\
- 基于正则,实现通用的日志解析工具,将日志文件处理为带Schema的结构化数据 \\
- 实现专用ETL模块,抽象日志元数据生成可以直接分析入库的数据集,统一数据源管理 \\
- 构建数据分析模块,实现常用的阈值算法(极值,四分位等)对异常数据进行分析报警 \\
- 实现基于Phantomjs的服务端图表生成模块,保证最佳的图表生成质量 \\
\\
\textbf{舆情监控系统-天兵} C\#, Node.js, Phantomjs 2012/07-2014/12 \\
- 实现通用爬虫系统(队列,排重,正文提取,存储,分布式,任务管理及任务指派等) \\
- 设计并实现舆论分析系统(贝叶斯分类,实时图表,舆情数据可视化,语音及邮件报警等) \\
- 开发Web信息采集器,带领团队完成特定信息的采集和分析,为竞品分析提供基础数据源 \\
- 实现通用网页正文提取算法:\url{https://github.com/stanzhai/Html2Article} 被一些公司采用 \\
\\
\textbf{济宁市基层党组织信息管理系统}  2011/10-2011/12 \\
本系统是结合了GIS的信息管理系统,地图部分采用ArcGIS开发,采用WebService \\
与网站通信获取党组织信息。实现了信息录入,信息管理,复杂的信息动态查询, \\
查询结果导出Excel,管理员权限管理及分配等核心模块。\\
技术实现:ASP.NET MVC3,Sql Server,NHibernate,ActiveRecord, \\
Spring.NET,Node.js,Lucene.net,Log4net,Nginx \\
\\
\textbf{Android五棍棋游戏}  2010/10-2010/11 \\
五棍棋是一款深具中国传统文化特色的棋类游戏; \\
独立负责Android版整个项目的开发(1W+代码量); \\
采用AndEngine游戏引擎实现,基于SOLID设计原则,分离了走棋算法、游戏交互、UI渲染模块; \\
通过剪枝优化了AI核心算法,实现了蓝牙对战、人机对战; \\
参加移动MM百万青年创业大赛,全国入围(比赛延期出了幺蛾子) \\
\\
\textbf{通用浏览器插件框架}  2009/06-2009/09 \\
目前(2009年)通用浏览器的引擎基本分为三种类型:WebKit、Gecko以及Trident等。 \\
这些不同的浏览器都有不同的插件标准,开发人员不得不为各种不同的浏览器开发同一种类型的插件。 \\
采用了面向对象设计思想,使用C++语言定义一套独立的插件标准,为不同引擎的浏览器提供了相应的 \\
适配器。通过使用本浏览器插件框架,开发插件的用户利用我们的标准只需开发一次插件,就能实现 \\
在多个浏览器上使用。降低浏览器插件开发成本,提高了开发效率和易维护性。 \\
- 基于Win平台C++动态库、多线程技术,抽象了接口规范(load,exec,unload核心函数) \\
- 针对IE浏览器插件、mozilla扩展开发库规范,实现了不同平台的框架容器 \\
- 基于接口规范实现了俄罗斯方块游戏demo,作为插件直接运行于不同浏览器 \\
- 本作品参加了山东省齐鲁软件大赛,并获一等奖。 \\

\section{\sc 其他}
\textbf{GitHub:} \url{https://github.com/stanzhai} \\
\textbf{专栏:} \url{https://zhuanlan.zhihu.com/bigdata-spark} \\
\textbf{Blog: } \url{https://stanzhai.site} \\
\textbf{折腾: } \url{https://www.zhihu.com/question/51314788/answer/144324074} \\

\end{resume}
\end{document}
